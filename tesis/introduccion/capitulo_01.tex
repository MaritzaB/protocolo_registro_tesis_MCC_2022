\section*{Introducción}
\subsection*{Antecedentes}

Los modelos de nicho ecológico (SDM por sus siglas en inglés, \textit{species
distribution models}) permiten hacer inferencias sobre la distribución
de especies y comunidades de especies, permitiendo identificar áreas
prioritarias para la conservación \cite{flores2019ecological}, en este caso de
aves marinas.

Un nicho ecológico es el rango de condiciones ambientales bajo las cuales un
individuo en particular tiene un éxito reproductivo de por vida, es decir las
condiciones idóneas bajo las cuales es capaz de reproducirse
\cite{takola2022hutchinson}.

Particularmente, las aves marinas son de importancia mundial ya que juegan un
papel importante en las funciones ecológicas de los sistemas marinos, lo que las
convierte en excelentes indicadores de largo plazo y gran escala
\cite{furness1997seabirds}.

De acuerdo con la Unión Internacional para la Conservación dela Naturaleza
(Birdlife International, 2021b), cerca de una tercera parte de las 368 especies
de aves marinas reconocidas se encuentran como críticamente amenazadas,
amenazadas o vulnerables, y el 47\% de todas las especies de aves marinas
mostraban tendencias de decrecimiento en sus poblaciones \cite{dias2019threats}.

En México habítan alrededor del 11\% de las aves de todo el mundo, con lo que
nuestro país se coloca en el onceavo lugar a nivel mundial de riqueza
avifaunística \cite{mendez2021ecologia}.

De las 368 especies de aves marinas del mundo, México alberga 126 especies
reconocidas, de las cuales 20 se encuentran globalmente amenazadas, cuatro
amenzadas y 13 como vulnerables \cite{mendez2021ecologia}.

Debido a esto, la importancia de establecer modelos ecológicos que nos permitan
delimitar áreas prioritarias para la conservavión de estas especies en nuestro país.

Los modelos de nicho ecológico utilizan datos de ocurrencias de alguna especie
que junto con datos ambientales nos permiten modelar correlativamente las
condiciones ambientales bajo las que vive la especie en cuestión y así nos
permiten predecir su hábitat idóneo.

Los modelos de nicho ecológico se usan de cuatro formas \cite{warren2011ecological}:

\begin{enumerate}
\item Para estimar la idoneidad relativa del hábitat en áreas geográficas que se
sabe que está ocupada por la especie.
\item Para estimar la idoneidad relativa del hábitat en áreas geográficas que no
se sabe que está ocupada por la especie.
\item Para estimar los cambios en la idoneidad del hábitat a lo largo del
tiempo dado un escenario específico de cambio ambiental.
\item Como estimación del nicho de la especie.
\end{enumerate}


\subsection*{Descripción de como se afronta esta problemática en la actualidad}

La regresión logística, regresión de componentes y el análisis de árboles de
clasificación y regresión, son comúnmente utilizados en el modelado ecológico
mediante SIG \cite{munoz2004comparison}. Sin embargo, es importante hacer una
comparativa de los algoritmos para evaluar cuál es el que nos genera un mejor
modelo de hábitat para aves marinas.


\subsection*{Problemática a abordar en la tesis}

Este trabajo busca resolver el problema de generar un marco de trabajo que nos
permita comparar los algoritmos de aprendizaje máquina utilizados para modelar
el nicho ecológico de aves marinas.

\section*{Características y generalidades}
\subsection*{Descripción de la solución al problema planteado}

Implementaremos tres algoritmos de aprendizaje máquina para generar modelos de
nicho ecológico a partir de variables oceanográficas. 

\subsection*{Justificación}

Los modelos de nicho ecológico nos permitirán determinar las características
ambientales del hábitat idóneo para alguna ave marina en específico.

\section*{Objetivos}
\subsection*{Objetivo General:}
Implementar diferentes algoritmos de aprendizaje máquina para modelado de nicho
ecológico a partir de variables oceanográficas.

\subsection*{Objetivos específicos:}

\paragraph*{Objetivo 1.}

Implementar diferentes algoritmos de aprendizaje máquina. Entre los
propuestos están: Regresión lineal, árboles de desición, algoritmos de
aprendizaje máquina como redes neuronales, máxima entropía o random forest.

\paragraph*{Objetivo 2.}

Generar modelos de nicho ecológico usando variables oceanográficas y datos
de desplazamientos de aves marinas.

\paragraph*{Objetivo 3.}

Comparar los algoritmos de aprendizaje máquina y evaluar cuál es el que
nos genera un mejor modelo.
