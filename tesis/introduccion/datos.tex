\section*{Datos}

\subsection*{Datos de la especie cuyo nicho ecológico se va a modelar}

Los datos utilizados para el modelado de nicho ecológico de este trabajo son
datos de las trayectorias de aves marinas, particularmente el Albatros de
Layssan \textit{Phoebastria immutabilis}.

Isla Guadalupe, frente a la península de Baja California, México, alberga a la
colonia reproductora más importante y en crecimiento de albatros de Laysan en el
Pacífico Oriental \cite{hernandez2014laysan}.

Los datos fueron tomados por el Grupo de Ecología y Conservación de Islas.
Otra fuente de datos a revisar es: \href{https://data.seabirdtracking.org/dataset/1928}{seabirdtracking}.

\subsection*{Datos de variables oceanográficas}

El modelado de nicho ecológico de aves marinas involucra el uso de variables
oceanográficas que son sabidas influyen en la distribución y abundancia de su
alimento. 

Los albatros de Layssan han demostrado patrones de reproducción en ambientes
olitogróficos, es decir, ambientes dónde los nutrientes esenciales como el
nitrógeno, fósforo y hierro están limitados. Durante la incubación y crianza,
los albatros de Layssan viajan a aguas más frías y productivas, pero están
restringidas al ambiente de baja productividad cerca de la colonia durante la
crianza, cuando los requerimientos son mayores \cite{kappes2015reproductive}.

Algunas de las variables oceanográficas comunmente usadas para el modelado de
nicho ecológico de albatros de Layssan son \cite{henry2021successful}:

\begin{itemize}
    \item Temperatura superficial del marco
    \item Concentración de clorofila
    \item Salinidad de la superficie del mar
    \item Corrientes oceanográficas
\end{itemize}


