
\section*{Datos}

\subsection*{Datos de la especie cuyo nicho ecológico se va a modelar}

Los datos utilizados para el modelado de nicho ecológico de este trabajo son
datos de GLS de las trayectorias de aves marinas, particularmente el albatros de
Layssan \textit{Phoebastria immutabilis}.

Isla Guadalupe, frente a la península de Baja California, México, alberga a la
colonia reproductora más importante y en crecimiento de albatros de Laysan en el
Pacífico Oriental \cite{hernandez2014laysan}.

Los datos fueron tomados por el Grupo de Ecología y Conservación de Islas. El
conjunto de datos contiene las trayectorias de albatros de Layssan de 47
individuos desde el año 2014 hasta el año 2018, distribuidos de la siguiente
manera: \\

\begin{table}[h!]
\begin{center}
\begin{tabular}{lcc}
    Año & Mes & Cantidad de trayectorias \\
    \hline
    \multirow{3}{*}{2014} & 05 & 2\\
    & 06 & 1\\
    & 12 & 4\\
    \hline
    \multirow{3}{*}{2015} & 01 & 4\\
    & 02 & 3\\
    & 03 & 2\\
    \hline
    \multirow{3}{*}{2016} & 02 & 2\\
    & 03 & 1\\
    & 04 & 1\\
    \hline
    \multirow{3}{*}{2017} & 02 & 6 \\
    & 03 & 5 \\
    & 08 & 1 \\
    \hline
    \multirow{2}{*}{2018} & 01 & 13 \\
    & 02 & 19 \\
    \hline
\end{tabular}
\caption{ Cantidad de trayectorias de individuos de albatros de Layssan
disponibles para el modelado}
\end{center}
\end{table}



Otra fuente de datos a revisar es:
\href{https://data.seabirdtracking.org/dataset/1928}{seabirdtracking}.

\subsection*{Datos de variables oceanográficas}

El modelado de nicho ecológico de aves marinas involucra el uso de variables
oceanográficas que son sabidas influyen en la distribución y abundancia de su
alimento. 

Los albatros de Layssan han demostrado patrones de reproducción en ambientes
olitogróficos, es decir, ambientes dónde los nutrientes esenciales como el
nitrógeno, fósforo y hierro están limitados. Durante la incubación y crianza,
los albatros de Layssan viajan a aguas más frías y productivas, pero están
restringidas al ambiente de baja productividad cerca de la colonia durante la
crianza, cuando los requerimientos son mayores \cite{kappes2015reproductive}.

Las variables oceanográficas que han demostrado tener una mayor influencia para
la predicción de nicho ecológico de albatros de Layssan durante la época de
reproducción y post-reproducción son \cite{henry2021successful}:

\begin{itemize}
    \item \textbf{Temperatura superficial del mar TSS (SST por sus siglas en
    inglés).} La TSS influye en la productividad biológica del océano y en la
    distribución de presas marinas. En áreas donde las aguas son más frías, como
    las corrientes ascendentes y las surgencias, se produce una mayor
    productividad biológica debido a la presencia de nutrientes ricos en el
    agua. Estas áreas atraen una mayor abundancia de presas, lo que a su vez
    atrae a los albatros de Laysan en busca de alimento.

    \item \textbf{Concentración de clorofila CCL (CHL por sus siglas en
    inglés).} De acuerdo con la NOAA (Oficina Nacional de Administracioń
    Oceánica y Atmosférica en español), los datos de CCL proveen un estimado de
    la biomasa de fitoplancton vivo en la capa superficial del mar. Este se
    utiliza como un indicador de productividad primaria, es decir, la cantidad
    de cantidad de materia orgánica que se produce a través de la fotosíntesis.
    Las áreas con una mayor CCL, como las regiones costeras, las surgencias y
    las zonas de afloramiento, suelen ser más productivas en términos de biomasa
    y abundancia de presas marinas.

    \item \textbf{Velocidad del viento.} Dado que los albatros de Laysan se
    caracterizan por desplazarse largas distancias, dependen del viento para
    moverse. La velocidad y dirección del viento influye en su capacidad de
    vuelo aunque también puede influir en la disponibilidad de alimento. Por
    ejemplo, aguas agitadas pueden dificultar la caza de peces y calamares.

\end{itemize}

La importancia de cada una de estas variables cambia con el lugar y periodo de
reproducción.

El conjunto de datos satelitales de estas variables están acotados a los datos
de trayectorias de albatros que tenemos.
